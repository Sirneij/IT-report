\chapter{Introduction to SIWES Program}

\section{Background}


The Student Industrial Work-Experience Scheme (SIWES) is an undergraduate training
programme with specific learning and career objectives geared towards the development of
occupational and industrial competencies of the participating students. It is a requirement for the
award of degrees and diploma to all students of tertiary institutions in Nigeria pursuing courses
in specialized engineering, technical, business, applied sciences and applied arts. The scheme is a
three-party program which involves the student, the tertiary institution, and the industry,and it exposes students to practical knowledge of their course of study.
Furthermore, SIWES is a general programme cutting across over 60 programmes in the
universities, over 40 programmes in the polytechnics and about 10 programmes in the colleges of
education. Thus, SIWES is not specific to any one course of study or discipline.
Consequently, the effectiveness of SIWES cannot be looked at in isolation with respect to a single
discipline, hence it is better explored in a holistic manner since many of the attributes, positive outcomes and challenges associated with SIWES are common to all disciplines participating in the scheme.
Hence, the approach of this report is to look at SIWES as a general study programme cutting across
several disciplines. Furthermore, the report gives details of the industrial work experience gained at ipNX Nigeria Limited, Victoria Island, Lagos state.
\section{Brief History of SIWES Program}


The Students’ Industrial Work-Experience Scheme (SIWES) started with about 748
students from 11 institutions of higher learning in 1974. By 1978, the scope of participation in the
scheme had increased to about 5,000 students from 32 institutions. The Industrial Training Fund,
however, withdrew from the management of the scheme in 1979 owing to problems of
organizational logistics and the increased financial burden associated with the rapid expansion of SIWES. Consequently, the Federal Government funded the scheme through the National Universities Commission (NUC) and the National Board for Technical Education (NBTE) who managed SIWES for five years (1979 – 1984). The supervising agencies (NUC and NBTE)
operated the scheme in conjunction with their respective institutions during this period
\section{Scope of SIWES}


The decree which legislates this scheme does not give a time confinement or restriction to
its implementation but it is a relative assignment. This implies that each school has the mandate
with respect to her academic calendar to categorically state when the scheme will be undertaken but must ensure that the required period of training is satisfied thus. 
\section{Aims and Objectives of SIWES}


The objectives of SIWES as stated in Information and Guideline for SIWES (2002) are:
\begin{enumerate}
	\item To provide an avenue for students in higher institutions to acquire industrial skills and experience in their approved course of study.
	\item To prepare students for the industrial works situation which they are likely to meet after graduation.
	\item To expose students to work methods and techniques in handling equipment and machinery
	not available in their institutions.

	\item To provide students with an opportunity to apply their knowledge in real work situation
	thereby bridging the gap between theory and practical.
	\item To enlist and strengthen employers’ involvement in the entire educational process and
	prepare students for employment in Industry and Commerce.
\end{enumerate}
