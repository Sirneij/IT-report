\chapter{Conclusion and Recommendation}
\section{Conclusion}
During this internship, I had the chance to work on many aspects of Software Engineering ranging from Planning and Requirement Analysis to Deployment. I developed new technical knowledge and soft skills while improving or enhancing previous ones in the course of implementing those robust projects.
\begin{itemize}
	\item I gained new knowledge in the area of Working with PostgreSQL - for full-text search capabilities - and SQLite Databases, regarding the various issues involved and mechanisms in these systems.
	\item I brushed up my knowledge of Python and JavaScript, as they were required to implement the projects.
	\item It was a challenging but awesome experience deploying applications to a Linux server from the ground up. Countless times of breaking and fixing server dependencies were worth the stress after all as the process exposed nifty things about systems administration.
	\item The importance and huge impact portability and scalability have on decisions about hardware, technologies/tools to be used, system architecture, algorithms, and so on were duly learnt and observed.
	\item Extensive exposure to Git, as a version control system, for safe storage of codes and projects.
	\item Significant knowledge of using asynchronous programming for some background tasks to provide better user experience was well learnt. 
\end{itemize}
No doubt, some theoretical knowledge gained in school proved helpful and useful such as CPE407(Software Development Techniques), a useful ingredient in following along with the software development processes at \textit{ip}NX Nigeria Limited, and CPE202(Programming in C Language) as well CPE302(Object Oriented Programming using C\#), for laying a good foundation in programming to pick up other languages effortlessly.
\section{Recommendation}
There is no gainsaying the fact that the idea of Industrial Training is laudable and its role in equipping students with relevant and soft skills cannot be overemphasized. Out of the various things learned and experiences gained, the undermentioned are worth taking into consideration:
\begin{enumerate}
	\item \textbf{Adoption and Enforcement of Project Based Learning (PBL)}: Project Based Learning (PBL) is a pedagogy in which students learn by actively engaging in real-world and personally meaningful projects. In this method, students work on a project over an extended period of time – from a week up to a semester – that engages them in solving a real-world problem or answering a complex question.
	
	As a result, students develop deep content knowledge as well as critical thinking, collaboration, creativity, and communication skills. Project Based Learning unleashes a contagious, creative energy among students and teachers. Using and enforcing this method, with proper and decent design, will help narrow the wide gap between University Education and Industry required knowledge.
	\item \textbf{Incorporation of a full course on Algorithms and Data structures}: Data structure and associated algorithms are an important part of any computer science or engineering curriculum as they expose students to critical and advanced thinking skills which will ultimately ameliorate problem-solving prowess. It is highly important that students are aware of them because most Interview questions are based on their application.
	\item \textbf{Use of innovative approaches to teaching and
		learning}: The use of innovative approaches to teaching and
	learning is very essential such as the use of teaching aids, ensuring the availability of teaching materials online before and after lectures among others.
	\item The Federal Government should make it an obligation for all firms to absorb students on Industrial training and ensure that such students are well-trained. This stretches out of how daunting and herculean it is to secure a placement by most students.
\end{enumerate}